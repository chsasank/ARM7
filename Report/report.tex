\documentclass[a4paper]{article}


\title{ARM7TDMI Processor Implementation in Verilog}

\author{Sasank Chilamkurthy, Varun Bairaboina, Bhanu Vikas, Nithin Nethipudi}

\date{May 2014}
\begin{document}
\maketitle

\begin{abstract}
We have implemented pipelined version of ARM v7 processor.
\end{abstract}



\section{Pipeline stages design}
We have 6 pipeline stages: 

\begin{enumerate}
	\item Instruction Fetch (F) 
	
	\item Register Read (R)

	\item Multiplier (M)

	\item ALU 

	\item Memory (MEM)

	\item Writeback (W)

\end{enumerate}

	\subsection{Instruction Fetch (F)} 
		We fetching instructions from instruction cache and determine type of instruction. There is not much complex logic here.
	
	\subsection{Register Read (R)} 
		We read all poosible registers that can be needed by a instruction.Also, forwarding decisions are made here.

	\subsection{Multiplier (M)}
		Multiplier block does any possible multiplications required.

	\subsection{ALU} 
		This stage has ALU, Barrel shifter and condition field checker.
		Flag outputs from ALU is forwarded to the same stage
		i.e. fed back to itself.

	\subsection{Memory (MEM)} 
		Accesses memory for Load/store instructions.

	\subsection{Writeback} 
		We write back to registers and PC depending on instruction.

\section{Implementation details}

We made different modules for each blocks mentioned above and instantiated them in top level controller module. We maintained a \texttt{will\_this\_be\_executed} signal at each pipeline stage to facilitate easier flushing and bubble insertion. Code is loaded by editing \texttt{initial} block of instruction cache module. We checked design with the standard fibonacci sequence generator program.

\end{document}